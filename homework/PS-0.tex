\documentclass{article}
\usepackage{soto-homework}

% set up whether we are printing assignment or solution
\newif\ifsolution
\input{solution}

\begin{document}

\chead{ENSP 438 Homework 0}
\chead{Due Date:  Thursday 16 Jan 2014}
\hrule
\vspace{10pt}

The problem set you turn in should be both legible and the methods and
assumptions you are using must be clear.  I strongly suggest that you
use diagrams where appropriate and clearly state your assumptions.
Typesetting of problem sets can increase the readability.  The homework
will be graded on how well you communicate your method as much as the
correctness of the result.

This is an ungraded problem set to see where your background skills are.
Please answer these questions to the best of your ability.  The results
will help me adapt my teaching to the class's background.  Please also
show enough work for me to see your thought process.

\problem{Previous Classes}

Please list the mathematics, physics, chemistry, and ENSP classes you
have taken so far.

\problem{Trigonometry}

\paragraph{a.} If $\phi$ equals 30 degrees and $a$ equals 10, what is the value of
$$ a\sin\phi\cos\phi $$

\solution{
Substitute given values and evaluate
$$ 10 \sin 30 \cos 30
= 10 \cdot 0.5 \cdot 0.866
= 4.33 $$
}

\paragraph{b.} Given the following triangle where $O=4$ and $A=3$, what is the value of
$\sin\theta$?  What is the value of $\theta$ in degrees?  What is the
value of $\theta$ in radians?

\includegraphics[width=2in]{./figures/right-triangle.jpg}

\solution{
We can find $H$ by the pythagorean theorem and then find that
$$ \sin\theta = 4/5 = 0.8 $$
Using the inverse sin function
$$ \theta = \sin^{-1}(0.8) = 53\ deg$$
We can then convert this to radians
$$ 53\ deg \frac{3.14\ rad}{180\ deg} = 0.925\ rad $$
}


\problem{Discounting}

If $r = 0.02$ and $n = 10$, what is the value of
$$ \frac{1}{(1 + r)^n} $$

\solution{
Substitute given values and evaluate
$$ \frac{1}{(1 + 0.02)^{10}}
= \frac{1}{(1.02)^{10}}
= \frac{1}{1.22}
= 0.82 $$
}

\problem{Power}

A entertainment center draws an average of 150 watts and is on for 12 hours.  How
many kilowatt-hours of energy are consumed?

\solution{
Energy is equal to the power multiplied by the time.
$$ 150\ W \cdot 12\ hours = 1800\ Wh = 1.8\ kWh $$
}

\problem{Ohms Law}

You have a 2000 ohm resistor and a 3 volt power supply.  If you connect
the resistor to the power supply, how much current flows?  How much
power is dissipated in the resistor?

\solution{
We can find the current from ohms law.
$$ V = IR $$
$$ I = V/R = 3\ V / 2000\ ohms = 0.0015\ amps = 1.5\ mA $$
The power can be found either by $P=I^2 R$ or $P=V^2/R$
$$ P = V^2/R = (3\ V)^2 / 2000\ ohms = 0.0045\ W = 0.0045\ mW $$
}

\problem{Spreadsheet}

Using your Seawolf google docs account, create an Google Docs
spreadsheet entitled "lastname firstname homework 0".  In the first
column of the spreadsheet, create the numbers 0 through 10.  In the
second column, create a formula that multiplies each of these numbers by
two.  Share the spreadsheet with me at my address (sotod@seawolf).

\problem{Time Spent}

Please estimate the amount of time you spent on this homework.

\end{document}
