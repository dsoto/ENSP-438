\documentclass{article}
\usepackage{soto-homework}

% set up whether we are printing assignment or solution
\newif\ifsolution
\input{solution}

\begin{document}

\chead{ENSP 438 Homework 1}
\chead{Due Date: Tuesday 11 Feb 2014}
\hrule
\vspace{10pt}

This problem set will be submitted in two parts.

Part is written, which can be submitted either on paper or
electronically.  The homework will be graded on how well you communicate
your method much more than the correctness of the result.  A correct
result with no justification will receive no credit.  I strongly
suggest that you use diagrams where appropriate and clearly state your
assumptions.

The other part is a spreadsheet.  You will turn in a single spreadsheet
with multiple pages or sheets and share it with me on Google Docs using
my Seawolf account.

{\tiny Last modified: \today}

\problem{Loan Rate}

Calculate the monthly payment for a 2014 Prius with an MSRP of \$24,200
with a 5 year loan at 8\%.  Does the monthly payment match your
intuition?  Does it match the payment you can calculate online at car
loan sites?

Explain how you have to adjust the formula for the CRF in order to get
the correct payment.

Write out the formula mathematically and show your work.

Written, no spreadsheet.

\solution{
We use the CRF formula to calculate the value of the payment that will
have the same net present value as the initial loan amount.  Since this
is a monthly payment, we must convert the yearly interest rate to a
monthly interest rate by dividing by 12.  We must also find the number
of periods of the loan by multiplying the 5 years by 12 and find that
there are 60 periods.

$$CRF = \frac {i(1+i)^n}{(1+i)^n-1}$$

Where $i=0.08 / 12= 0.00667$ and $n = 5 \cdot 12 = 60$.

$$ CRF = 0.020276 $$

$$ payment = CRF * initial amount = 0.020276 * 24200  = 490.69 $$

}

\problem{Credit Card Balance}

Assume you have a credit card with a 15\% yearly interest rate and you
make a \$1000 purchase.  What is the balance after 6 months if you do
not make any payments?

Hint: this is the opposite of discounting to find the present value,
instead we are trying to find the future value.  The math will look
similar but has a key difference.

Written, no spreadsheet.

\solution{

The initial balance is compounded six times at the monthly rate.  We
find the monthly rate by dividing the annual rate (15\%) by 12.

$$ FV = (1 + i_y/12)^6 \cdot PV$$
$$ FV = (1 + 0.15/12)^6 \cdot \$1000 $$

$$ FV = \$1077.38 $$

}


\problem{Solar Loan}
Assume we want to take out a loan to buy a large (100 kW) solar system
which will cost \$2.10 per watt to install everything.  Assume the
panels will be installed in an area where each kW of solar panels will
generate 1500 kWh of solar electricity throughout the year and that we
can sell all the electricity we generate at \$0.10 per kWh.

Assuming that the solar panels will last 20 years and we can get a 20
year loan, what is the highest loan rate that we can accept to build
this project and still have positive cash flow each year?

Turn in both a written part explaining your methods and a spreadsheet
sheet named ``Solar''.

\solution{
My approach to this problem is to create a spreadsheet that has the
cash flows generated by the solar panel.  I'll then use the spreadsheet
to calculate the Internal Rate of Return of this solar investment.

My spreadsheet calculates a value of about 3\% for the IRR for this
investment.

The initial investment is \$210,000.  If I take out a loan at an
interest rate greater than 11.363\%, the monthly payment to the bank
will exceed the monthly income from the electricity sold.

}


\problem{Lighting Net Present Value}

A business is deciding whether or not to replace its light bulbs.  The
first option is to continue using incandescents, which must be replaced
every year.  Assume both bulbs and electricity are paid for at the
end of the year.  Assume 10 years.

Written: Explain your general approach and how you will create the
calculation.  Also comment on anything interesting you see in the
results.  Also summarize your results and make a recommendation.

Create a spreadsheet that performs a net present value comparison
between the cost of the incandescent vs the LED light bulb.  Turn in two
sheets, one named ``Incandescent NPV'' the other, ``LED NPV''.

Assumptions:

\begin{tabular}{l c}
Discount rate               & 5\% \\
Incandescent bulb cost (\$) & 1.00 \\
LED bulb cost (\$)          & 15.00 \\
Hours per day of lighting   & 6 \\
Cost of electricity         & 0.15 \\
Days per year of lighting   & 365 \\
Incandescent power          & 60W \\
LED power                   & 10W \\
Incandescent bulb lifetime  & 1 years \\
LED bulb lifetime           & 5 years \\
\end{tabular}

\solution{

The net present value of all the costs I have to pay for the LED
lighting are \$7,379.49 while the incandescent net present value is
\$29,820.24.  The net present value of all the money for LED lighting is
much less than that for incandescent lighting.  I would then choose the
LED lighting.

It is interesting that the cost of buying LED bulbs is much greater, but
must be paid less often.  It is also interesting how much greater the
incandescent electricity cost is.

}

\problem{Time Spent}

Please estimate the amount of time you spent on this homework.


\end{document}
