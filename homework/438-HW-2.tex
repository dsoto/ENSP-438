\documentclass{article}
\usepackage{soto-homework}

% set up whether we are printing assignment or solution
\newif\ifsolution
\input{solution}

\begin{document}

\chead{ENSP 438 Homework 2}
\chead{Due Date:  Thursday 20 Feb 2014}
\hrule
\vspace{10pt}

% HW2 time spent
% 28 jan, 0.5

This problem set will be submitted in two parts.  Part is written, which
can be submitted either on paper or electronically.  The homework will
be graded on how well you communicate your method much more than the
correctness of the result.  A correct result with no justification will
receive no credit.  I strongly suggest that you use diagrams where
appropriate and clearly state your assumptions.

The other part is a spreadsheet.  You will turn in a single spreadsheet
with multiple pages or sheets and share it with me on Google Docs using
my Seawolf account.  Please place all spreadsheets in a single file
named "Homework 2".  You can begin the spreadsheet in Excel on your own
computer and then upload it to Google Docs.  When you upload, please
convert the file to a Google Doc and make sure I have both read and
write permission.  This allows me to write feedback directly in the
spreadsheet and see your formulas.

{\tiny Last modified: \today}


\problem{REEPS2}

\subproblem{REEPS2 Problem 1.1}


\solution{

The key is to understand the relationship between the efficiency of the
plant and the heat rate.

$$ \eta = \textrm{efficiency} = \ufrac{electrical energy delivered}
{chemical energy consumed} $$



The efficiency of the plant has no units since the energy on the top and
bottom are expressed in the same units.

The heat rate has units since chemical energy is often expressed in BTU
or kJ and electrical energy is often expressed in kWh.

$$ \textrm{heat rate} = \ufrac{chemical energy consumed (BTU or kJ)}{electrical
energy delivered (kWh)} $$

We can convert the unit-less efficiency to a heat rate by using a unit
conversion.

$$ \ufrac{1 kWh chemical energy}{0.52 kWh electrical energy} \cdot
\ufrac{3600 kJ chemical energy}{1 kWh chemical energy}
= 6923 \ufrac{kJ chemical energy}{kWh electrical energy} $$

$$ \ufrac{1 kWh chemical energy}{0.52 kWh electrical energy} \cdot
\ufrac{3412 BTU chemical energy}{1 kWh chemical energy}
= 6562 \ufrac{BTU chemical energy}{kWh electrical energy} $$

From the heat rate in kJ/kWh, we can perform a unit conversion since the
amount of carbon and carbon dioxide released are proportional to the
amount of chemical energy consumed.

$$ \ufrac{6923 kJ natural gas}{1 kWh electrical energy} \cdot
\ufrac{1 kg natural gas}{55,340 kJ natural gas} \cdot
\ufrac{0.77 kg carbon}{1 kg natural gas}
= 0.096 \textrm{kg C/kWh} $$

To get the carbon dioxide, we further multiply by the ratio of the
weights of carbon dioxide and carbon (44/12 or 3.67)

$$ \ufrac{6923 kJ natural gas}{1 kWh electrical energy} \cdot
\ufrac{1 kg natural gas}{55,340 kJ natural gas} \cdot
\ufrac{0.77 kg carbon}{1 kg natural gas} \cdot
\ufrac{44 kg carbon dioxide}{12 kg carbon}
= 0.353 \textrm{kg C/kWh} $$

The coal plant in Example 1.1 has a carbon emission rate of 0.267
kgC/kWh.  This natural gas plant has a carbon emission rate about 1/3 of
this.


%\includegraphics{./REEPS2-Solutions/R2_1_1.pdf}
}


\subproblem{REEPS2 Problem 1.3}

% for next year, what is the most important difference between coal and
% natural gas, the heat rate or the carbon intensity of fuel?  you may
% also show the calculation once in written form and perform the other
% calculations in a spreadsheet if you wish.

\solution{

The solution from the book misses the conversion from kgC to kgCO$_2$.
Each of these numbers should be multiplied by 44/12 (the conversion from
mass carbon to mass carbon dioxide) to get the added
carbon cost on the electricity.

We can solve this as a unit conversion problem.  The key insight is that
the tax, carbon emitted, and fuel burned per kilowatt hour are all
proportional.  Be sure you understand what each number and conversion
factor represents.  It is important to distinguish between chemical and
electrical energy.

The heat rate gives us the amount of chemical energy combusted for every
unit of electrical energy generated.

$$ \textrm{Heat Rate} = \ufrac{Chemical Energy Consumed}{Electrical Energy
Created} $$

We then multiply this by the chemical energy density, rate of carbon
dioxide emissions per unit fuel, tax rate per unit carbon dioxide.  Here
is the main idea.


$$ \ufrac{Chem Energy}{Elec Energy} \cdot
\ufrac{Fuel Mass}{Chem Energy} \cdot
\ufrac{Carbon Dioxide Mass}{Fuel Mass} \cdot
\ufrac{Tax Cost}{Carbon Dioxide Mass}
= \ufrac{Tax Cost}{Elec Energy}$$

To get the numbers right, apply the appropriate unit conversions.

a) Old Coal (heat rate 10,500 BTU/kWh, carbon content 24.5 kg carbon/GJ)

$$ \ufrac{10,500 Btu}{1 kWh} \cdot
\ufrac{1055 J}{1 Btu} \cdot
\ufrac{24.5 kg carbon}{10$^9$ J Fuel} \cdot
\ufrac{44 kg CO$_2$}{12 kg carbon} \cdot
\ufrac{50 USD}{1000 kg CO$_2$} \cdot
\ufrac{100 cents}{1 USD} =
4.97\ cents / kWh $$

b) New Coal (heat rate 8,500 BTU/kWh, carbon content 24.5 kg carbon/GJ)

$$ \ufrac{8,500 Btu}{1 kWh} \cdot
\ufrac{1055 J}{1 Btu} \cdot
\ufrac{24.5 kg carbon}{10$^9$ J Fuel} \cdot
\ufrac{44 kg CO$_2$}{12 kg carbon} \cdot
\ufrac{50 USD}{1000 kg CO$_2$} \cdot
\ufrac{100 cents}{1 USD} =
4.03\ cents / kWh $$

c) Integrated gasification, combined cycle coal (heat rate 9,000
BTU/kWh, carbon content 24.5 kg carbon/GJ)

$$ \ufrac{9,000 Btu}{1 kWh} \cdot
\ufrac{1055 J}{1 Btu} \cdot
\ufrac{24.5 kg carbon}{10$^9$ J Fuel} \cdot
\ufrac{44 kg CO$_2$}{12 kg carbon} \cdot
\ufrac{50 USD}{1000 kg CO$_2$} \cdot
\ufrac{100 cents}{1 USD} =
4.25\ cents / kWh $$

d) Natural Gas combined cycle (heat rate 7,000 BTU/kWh, carbon content
13.7 kg carbon/GJ)

$$ \ufrac{7,000 Btu}{1 kWh} \cdot
\ufrac{1055 J}{1 Btu} \cdot
\ufrac{13.7 kg carbon}{10$^9$ J Fuel} \cdot
\ufrac{44 kg CO$_2$}{12 kg carbon} \cdot
\ufrac{50 USD}{1000 kg CO$_2$} \cdot
\ufrac{100 cents}{1 USD} =
1.87\ cents / kWh $$


e) Natural Gas combustion turbine (heat rate 9,500 BTU/kWh, carbon
content 13.7 kg carbon/GJ)

$$ \ufrac{9,500 Btu}{1 kWh} \cdot
\ufrac{1055 J}{1 Btu} \cdot
\ufrac{13.7 kg carbon}{10$^9$ J Fuel} \cdot
\ufrac{44 kg CO$_2$}{12 kg carbon} \cdot
\ufrac{50 USD}{1000 kg CO$_2$} \cdot
\ufrac{100 cents}{1 USD} =
2.52\ cents / kWh $$

The natural gas plants have less possible tax liability by about a
factor of 2.  Note that the natural gas heat rates are not lower than
the coal heat rates but the carbon released per amount of chemical
energy is lower.  Also note that this calculation is redundant and a
good candidate for a spreadsheet.

%\includegraphics{./REEPS2-Solutions/R2_1_3.pdf}
}



\subproblem{REEPS2 Problem 1.11}


\solution{

The key to this problem is understanding when the coal plants are run
and what the average capacity factor means.  Since coal plants are
usually most economical when run constantly, the utility will try to
keep them on all the time.  If the load is 600MW or below, this utility
will only run coal plants.

For the capacity factor of the entire fleet of coal plants you have to
consider two quantities.  First, the amount of electrical energy created
by all the coal plants.  Second, the amount of electrical energy that
could have been created if all 600MW of the coal plants were run all
year long.  The ratio of these two numbers is the capacity factor.

\includegraphics{./REEPS2-Solutions/R2_1_11ac.pdf}

The capacity factor is best expressed for this example as

$$ CF = \ufrac{Electrical energy actually delivered by coal plants}
{Coal electrical energy if plants were run all year} $$

The energy actually delivered is the shaded area since coal supplies
100\% of the demand below 600 MW.  The possible energy is the 600 MW
times the 8760 hours in a year.

\includegraphics{./REEPS2-Solutions/R2_1_11d.pdf}

The energy delivered by the coal plants is the top part of the capacity
factor.

}



\problem{Solar Levelized Cost}

Using your solar example from Homework 1.  If you finance the plant with
a 20-year loan at 5.0\%, what will your levelized cost of energy be?

Create a spreadsheet showing the cash flows over the 20 year life and
use it to make your calculation.  Remember, levelized cost of energy is
the total cost per year divided by the energy sold per year.

What will the levelized cost of electricity be if the solar farm can
generate and sell 1500 kWh for every kW installed?  1200 kWh? 1800 kWh?
You should be able to change one cell on your spreadsheet to find these
values.

\solution{

See the linked spreadsheet solution.

The key is to understand what fixed and variable costs we have for this
simplified solar plant.  The fixed cost is the yearly payment on the
loan while we ignore any variable costs in this example since sunlight
is available at no cost.  We then calculate the levelized cost by
dividing these costs by the amount of electrical energy we can generate
and sell.

My spreadsheet shows a cost of 0.14, 0.11, 0.09 USD per kWh for 1200,
1500, and 1800 kWh per year per kW installed.

You may notice that it is not necessary to show all 20 rows representing
20 years to do this calculation.  We will be using these in the future.
}


\problem{Diesel Levelized Cost}

In an off-grid village, diesel generators are often used.  We want to
calculate the cost of providing electricity using a diesel generator.
If our diesel genset costs \$500/kW and uses 0.35 liters of gasoline to
generate a kWh of electricity.  Assume a 10 year loan at 5.0\% interest.
Assume the generator runs for 6 hours each day.  Assume the generator is 5
kW and the village demand during the 6 hours is exactly 5 kW.

Create a spreadsheet showing the cash flows over the life of the plant
and make the levelized cost calculation.  Create a column for the
finance cost and the fuel cost.

Use your spreadsheet to calculate the electricity cost at \$0.75, \$1.0,
and \$1.25 per liter.  Also calculate the electricity cost for a 5kW and
10 kW generator.

Comment on anything you find interesting or surprising.


\solution{

See the linked spreadsheet solution.

Comments:

0.29, 0.38, 0.47 USD per kWh for 0.75, 1.00, and 1.25 USD per liter.

Because all the costs in our spreadsheet are proportional to the size of
the generator, you will not see any change in the cost of electricity
when you change the size of the generator.  In practice larger
diesel generators have better LCOE than smaller generators.


The diesel cost of electricity is over three times the price of the
solar electricity.  However, the diesel engine can be run at any time of
day or night, which is an advantage that may outweigh the cost benefit.
We did not include in the solar levelized cost the cost of batteries to
use electricity at night.
}


\problem{Final Project}

List three possible topics for your final project.

\problem{Time Spent}

Please estimate the amount of time you spent on this homework.





\end{document}

\chead{ENSP 438 Homework 3}
\chead{Due Date:  Tuesday 11 Mar 2014}
\hrule
\vspace{10pt}

This problem set will be submitted in two parts.

Part is written, which can be submitted either on paper or
electronically.  The homework will be graded on how well you communicate
your method much more than the correctness of the result.  A correct
result with no justification will receive no credit.  I strongly
suggest that you use diagrams where appropriate and clearly state your
assumptions.

The other part is a spreadsheet.  You will turn in a single spreadsheet
with multiple pages or sheets and share it with me on Google Docs using
my Seawolf account.  Please place all spreadsheets in a single file
named "Homework 2".  You can begin the spreadsheet in Excel on your own
computer and then upload it to Google Docs.  When you upload, please
convert the file to a Google Doc and make sure I have both read and
write permission.  This allows me to write feedback directly in the
spreadsheet and see your formulas.

{\tiny Last modified: \today}

\problem{REEPS2 Problem 4.1}

Think about the angles allowed and restricted by the light shelf.

\problem{REEPS2 Problem 4.5}

\problem{REEPS2 Problem 4.12}

Instead of Excel, use Google docs.  Let me know if you have problems
downloading a CSV into Google docs.

\problem{Between Darwin and Stevenson}

SSU-Henge

When will the tip of the shadow of bacon and eggs just reach X.

What date will the shadow of Darwin reach the closest to Stevenson?  How
close will it get?

What does this mean for putting solar panels in between the two
buildings.

\problem{REEPS2 Problem 5.9}

\problem{What resistor will deliver the maximum power for a given IV
curve?}


\problem{Time Spent}

Please estimate the amount of time you spent on this homework.




\chead{ENSP 438 Homework 4}
\chead{Due Date:  Tuesday 25 Mar 2014}
\hrule
\vspace{10pt}

\problem{Time Spent}

Please estimate the amount of time you spent on this homework.



\chead{ENSP 438 Homework 5}
\chead{Due Date:  Tuesday 08 Apr 2014}
\hrule
\vspace{10pt}
\problem{Time Spent}

Please estimate the amount of time you spent on this homework.

\chead{ENSP 438 Homework 6}
\chead{Due Date:  Tuesday 22 Apr 2014}
\hrule
\vspace{10pt}
\problem{Time Spent}

Please estimate the amount of time you spent on this homework.

\end{document}
